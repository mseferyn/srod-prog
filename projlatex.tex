\documentclass{report}
\usepackage[polish]{babel}
\usepackage[utf8]{inputenc}
\usepackage[T1]{fontenc}
\usepackage{graphicx} 
\title{Ciąg Fibonacciego}
\author{Marcin Seferyn}
\date{12.12.2019}
\begin{document}
\maketitle
\newpage
\tableofcontents
\newpage
\chapter{Wstęp}
\section{Kim był Leonardo Fibonacci}
\cite{fib}
Leonardo z Pizy, urodzony ok. 1175 roku w Pizie - zmarł w 1250 roku. Był włoskim matematykiem, znany jako Leonardo Fibonaccim, Filius Bonacci oraz Leonardo Pisano
\subsection{Życiorys}
Jego ojciec, Guglielmo z rodziny Bonacci, zajmował stanowisko dyplomatyczne w Afryce północnej i Fibonacci tam właśnie się kształcił. Pierwsze lekcje matematyki pobierał od arabskiego nauczyciela w mieście Boużia (dziś algierska Bidżaja). Dużo podróżował najpierw razem z ojcem, później samodzielnie, odwiedzając i kształcąc się w takich miejscach jak Egipt, Syria, Prowansja, Grecja i Sycylia. W czasie swych podróży po Europie i po krajach Wschodu miał okazję poznać osiągnięcia matematyków arabskich i hinduskich, między innymi dziesiętny system liczbowy.
Około 1200 roku Fibonacci zakończył podróże i powrócił do Pizy.
Tam zajął się opracowywaniem wielu zagadnień starożytnej matematyki. Żył w epoce przed wynalezieniem druku, dlatego jego dzieła mogły zostać rozpowszechnione jedynie za pomocą ręcznego odpisu. Z tego powodu do dziś przetrwało jedynie kilka z jego prac.
Fibonacci zasłużył się dla rozwoju miasta. Jego wysiłki zostały docenione przez cesarza Fryderyka II. Zmarł w Pizie w 1250 roku.


\begin{figure}[h]
\center
\includegraphics[scale=0.6]{fibo}
\caption{Leonardo Fibonacci}
\end{figure}
\newpage
\section{Dokonania}
Fibonacci był niezwykle utalentowanym matematykiem, jednak wiele z jego prac i teorii nie oddziałało na rozwój matematyki, ponieważ pozostały w dużej mierze nieznane w okresie średniowiecza.
Jednym z ważniejszych dzieł Leonarda z Pizy było „Liber abaci”, które stanowiło wykład azjatyckich osiągnięć w dziedzinie matematyki. Pojawiły się tu takie pojęcia jak: liczby ujemne, zero, pozycyjny system zapisu liczby, równania liniowe i kwadratowe.
Ponadto ważną pracą była \textit{Practica geometriae}, gdzie Fibonacci po raz pierwszy użył algebry w dziedzinie geometrii.
\newpage
\begin{thebibliography}{9}
\bibitem{fib} Kim był Fibonacci https://pl.wikipedia.org/wiki/Fibonacci
\end{thebibliography}
\end{document}
	