\documentclass{article}
\usepackage[polish]{babel}
\usepackage[utf8]{inputenc}
\usepackage[T1]{fontenc}
\title{Ciąg Fibonacciego}
\author{Marcin Seferyn}
\date{12.12.2019}
\begin{document}
\maketitle
\newpage
\tableofcontents
\newpage
\section{Wstęp}
\subsection{Kim był Leonardo Fibonacci}
Leonardo z Pizy, urodzony ok. 1175 roku w Pizie - zmarł w 1250 roku. Był włoskim matematykiem, znany jako Leonardo Fibonaccim, Filius Bonacci oraz Leonardo Pisano
\subsection{Życiorys}
Jego ojciec, Guglielmo z rodziny Bonacci, zajmował stanowisko dyplomatyczne w Afryce północnej i Fibonacci tam właśnie się kształcił. Pierwsze lekcje matematyki pobierał od arabskiego nauczyciela w mieście Boużia (dziś algierska Bidżaja). Dużo podróżował najpierw razem z ojcem, później samodzielnie, odwiedzając i kształcąc się w takich miejscach jak Egipt, Syria, Prowansja, Grecja i Sycylia. W czasie swych podróży po Europie i po krajach Wschodu miał okazję poznać osiągnięcia matematyków arabskich i hinduskich, między innymi dziesiętny system liczbowy.
Około 1200 roku Fibonacci zakończył podróże i powrócił do Pizy. 
\newpage
\begin{thebibliography}{99}
\end{document}
	
